%%
%% Copyright (c) 2019-2020 ChenJiaGeng <mail@deocnf.xyz>
%% CC BY 4.0 License
%%
%% Created: 2019-04-11
%%

% Chinese version
\documentclass[zh]{resume}

% Adjust icon size (default: same size as the text)
\iconsize{\Large}

% File information shown at the footer of the last page
\fileinfo{%
  \faCopyright{} 2024--2024, 陈佳庚 \hspace{0.5em}
  \creativecommons{by}{4.0} \hspace{0.5em}
  \githublink{de-conf}{resume} \hspace{0.5em}
  \faEdit{} \today
}

\name{佳\ 庚}{陈}

\keywords{Linux, Programming, Python, C, Shell,Docker, SysAdmin, Ci,KVM}

% \tagline{\icon{\faBinoculars}} <position-to-look-for>}
% \tagline{<current-position>}

% \photo{<height>}{<filename>}

\profile{
  \mobile{18700668464}
  \email{mail@deconf.xyz}
  \blog{deconf.xyz} \\
  \university{西安石油大学}
  \degree{网络工程 \textbullet 本科}
  \birthday{1998-04-07} 
  \github{de-conf} \\
  \job{求职意向:Linux系统运维,云计算方向,Devops}
  \address{陕西省 \textbullet 西安市}
  % \icontext{<fa-briefcase>}{<text>}
  % Custom information:
  % \icontext{<icon>}{<text>}
  % \iconlink{<icon>}{<link>}{<text>}
}

\photo{64pt}{resume.jpg}
\begin{document}
\makeheader

%======================================================================
% Summary & Objectives
%======================================================================
{\onehalfspacing\hspace{2em}%
本人乐观向上,自我驱动力和学习能力强,热爱尝试新事物,认同开源和分享精神.
擅长,热衷计算机和网络技术,
有 5 年的 Linux 使用经验,掌握 Shell、Python 等脚本语言编程
\par}

%======================================================================
\sectionTitle{技能点}{\faWrench}
%======================================================================
\begin{competences}
  \comptence{操作系统}{%
    {\faLinux} Linux (5 年)---(CentOS, Ubuntu,Alpine)
  }
  \comptence{编程}{%
  {\faPython} Python,{\faTerminal} Shell, {\faDatabase} SQL
  }
  \comptence{运维}{%
  Ansible, {\faDocker}Docker, K8s
  }
  \comptence{监控}{%
    Zabbix,Prometheus,Grafana,ELK Stack
  }
  \comptence{工具}{%
    Git, Make, Tmux, Vim
  }
  \comptence{网络与安全}{%
    Wireshark,Firewalls,VPN,SSL/TLS
  }
  \comptence{数据库}{%
  Postgres,Mysql,Oracle,redis,MongoDB,Prometheus
  }
  %\comptence{\icon{\faLanguage} 语言}{
  %  \textbf{英语} --- 读写(基础)
  %}
\end{competences}

%======================================================================
\sectionTitle{教育背景 \&\& \ \ \icon{\faBriefcase}工作经历}{\faGraduationCap}
%======================================================================
\begin{educations}
  \education%
    {2021.06}%
    [2016.09]%
    {西安石油大学}%
    {计算机学院}%
    {网络工程}%
    {本科}
\end{educations}
\begin{educations}
  \education%
    {至今}%
    [2021.6]%
    {西安}%
    {新路网络科技有限公司}%
    {运维服务部}%
    {运维工程师}
\end{educations}
%======================================================================
\sectionTitle{专业认证}{\faStamp}
%======================================================================
\begin{itemize}
  \item \link{https://deconf.xyz/certification/aliyun_acp.jpg}{阿里云.云计算工程师 (ACP)}
\end{itemize}
%======================================================================
\sectionTitle{专业技能}{\faCogs}
%======================================================================
\begin{itemize}
  \item 精通Linux操作系统内核调优,numa架构调优,dpdk参数调优等.
  \item 精通docker容器构建,docker-compose容器编排.精通docker各类部署和运维经验.
  \item 熟悉各类zabbix,grafana,prometheus 监控配置与调优.
  \item 熟悉部署工具ansible. 熟悉各类CI/CD 工具,Jenkins构建pipeline脚本等.
  \item 精通shell/python脚本语言,熟练处理各类线上业务与数据.
  \item 掌握各种nginx,,tomcat,apache 等web中间件配置.
  \item 掌握各类虚拟化kvm, proxmox, vmware esxi虚拟化技术与配置.
  \item 掌握postgres,mysql,oracle 主流数据库操作与优化.
  \item 了解网络设备交换路由配置.精通wireshark分析网络,进行网络排障与调优.
\end{itemize}

%======================================================================
\sectionTitle{工作经验}{\faCode}
%======================================================================
\begin{itemize}
  \item 近 100+ 生产环境Linux服务器日常运维工作,以及处理突发应急事件.\\ 熟练处理各类Linux发行版(centos/debian/ubuntu/alpine)问题与故障. 
  \item 线上 2700+ 交换路由设备的监控与日志留存,精通zabbix监控模板的建立,修正与完善.\\精通zabbix 服务器与代理的维护与等.
  \item 生产环境elk与prometheus、granfa维护. 
  \item 产品线高性能网关的维护与运维. 熟悉Linux系统调优与网卡优化以及dpdk参数优化.
  \item 线上5+所高校认证系统的维护.熟悉3A流程,熟悉freeradius配置,熟悉unlang 配置语法.
  \item 线上生产环境认证系统的部署与上线.精通docker-compose 容器编排. 熟悉云平台AWS的日常运维管理. 
  \item 内网高性能网关的研发环境配置,掌握CI/CD流程,熟悉Jenkins构建流程。
  \item 内网公共平台 confluence, jira ,gitlab平台的搭建与维护.
  \item 各类型数据包的分析,熟练定位网络故障与性能瓶颈.
  \item 内部运维知识库的管理与构建,输出相关的培训文档与手册.
\end{itemize}


%======================================================================
\sectionTitle{项目经验}{\faBriefcase}
%======================================================================
\begin{experiences}
  
  \experience
  [至今]%
  {2021.06}%
  {高性能网关项目}%
  [\begin{itemize}
    \item 项目介绍:\\
    本项目主要基于\link{https://wiki.fd.io/view/VPP}{VPP} + \link{https://www.dpdk.org}{DPDK} 开源框架,构建了IPOE TO PPPOE代拨网关.
    \item 主要职责:\\
    负责项目的开发环境的搭建,性能的测试与评估,线上的维护与部署. 
    \item 主要业绩: \\
    使用systemd-coredump捕获异常崩溃,协助研发分析原因,并构建了CI/CD 流程, 实现了提交代拨后自动构建,自动打包的操作. \\
    使用 \link{https://trex-tgn.cisco.com/}{TREX} 构建各类数据包协助研发测试性能与输出指标. \\
    上线后的故障排查,日志的分析与包括分析各类数据包 pppoe会话的交互,radius 会话交互等.\\
    输出标准化部署流程文档,指导其他节点标准化的部署.并使用zabbix脚本监控运行时指标,确保运行的稳定性. 
  \end{itemize}]
  \\
  \experience
  [至今]%
  {2021.06}%
  {高校认证项目}%
  [\begin{itemize}
    \item 项目介绍:\\
    基于\link{https://freeradius.org/}{FreeRadius} 为基础,自研计费的认证系统.
    \item 主要职责:\\
    新增学校认证系统的部署与维护,与三方认证系统的对接(电信/移动/派网).满足运营提出的认证策略(使用freeradius unlang构建的policy).各高校网络设备的监控,修正并扩展zabbix监控模板,并对接钉钉告警.
    与第三方(BRAS)网关的对接(portal 协议的对接).
    \item 主要业绩: \\
    负责的高校(西北大学,西安科技大学,西北政法大学,西安欧亚学院,湖北经济学院)认证系统500+天无故障,运行平稳.
    对于华为私有的portal协议重新构建了\link{https://github.com/de-conf/huawei-portalv2-wireshark-pluging/blob/main/wireshark-plugin-huawei-portal-v2.lua}{portal协议wireshark解析脚本},方便协议分析与排错.
  \end{itemize}]
  \\
  \experience
  [2023.06]%
  {2022.06}%
  {出口缓存项目}%
  [\begin{itemize}
    \item 项目介绍:\\
    基于\link{https://lancache.net/}{nginx的缓存项目}的部署.减少出口带宽占用,加快内网下载速度.
    \item 主要职责: \\
    项目的选型,网络架构的设计. 
    \item 主要业绩: \\
    部署后节约出口带宽2Gbps,内网可达1000Mbps的下载速率,用户满意度达到95\%. 解决线上关键问题\link{https://github.com/lancachenet/monolithic/issues/175}{下载卡99\%问题}
  \end{itemize}]
  \\
  \experience
  [2023.09]%
  {2023.06}%
  {内部各系统维护改造}%
  [\begin{itemize}
    \item 项目介绍:\\
    公司内部系统维护与改造,包括confluence,jira,gitlab等.
    \item 主要职责: \\
    内部系统硬件选型到部署,旧系统数据迁移等.由物理裸机部署到docker 容器化部署等。
    \item 主要业绩: \\
    系统稳定性、可维护性提升,由经常性宕机到目前稳定运行2年. 可维护性提升,容器化部署后,系统可升级,可迁移,可扩展.
  \end{itemize}]
  \\
  \experience
  [2024.01]%
  {2023.09}%
  {zabbix 监控系统优化}%
  [\begin{itemize}
    \item 项目介绍:\\
    生产环境zabbix监控系统优化,包括监控模板的修正,告警的优化等.
    \item 主要职责: \\
    zabbix系统性能优化,数据库优化等.
    \item 主要业绩: \\
    zabbix 由旧版本5.4 升级为最新版本 7.0.\\
    后端数据库由mysql 替换为timescaledb(基于postgres)的时序数据库,添加定时分表策略,性能提升500\%。通过压缩策略,数据库空间占用减少200\%. \\
    按照网络设备型号,通过厂商mib文件,构建了新的监控模板,单机监控项达到1000+,告警规则达到100+,告警准确率达到99.9\%。
  \end{itemize}]
  \\
  \\
  \experience
  [2024.02]%
  {2024.01}%
  {域名证书的维护}%
  [\begin{itemize}
    \item 项目介绍:\\
    生产环境域名证书的维护,包括证书的申请,更新,部署等.
    \item 主要职责: \\
    证书的申请,更新,部署等.
    \item 主要业绩: \\
   大量域名证书使用certbot自申请,自动更新,自部署,减少人工干预,提高效率,减低人工失误. \\
   全部域名增加监控,监控证书的过期时间,提前30天告警,保证证书的有效性.
  \end{itemize}]
  \\
  \experience
  [至今]%
  {2024.01}%
  {AWS 云平台维护}%
  [\begin{itemize}
    \item 项目介绍:\\
    生产环境AWS云平台维护,包括EC2,CloudFront,S3等.
    \item 主要职责: \\
    稳定性维护,成本优化,安全性优化等.
    \item 主要业绩: \\
    云平台管理, EC2 实例规格的优化,由按需实例转为预留实例,成本节约30\%. 设置EC2 默认的安全组,避免误操作导致的安全问题. \\
    使用AWS的cloudwatch,设置监控项,监控实例的CPU,内存,磁盘等,保证实例的稳定性. \\
    使用AWS的S3,设置数据备份策略,保证数据的安全性.
  \end{itemize}]
  \\
  % \experience
  % [2021.05]%
  % {2021.03}%
  % {在中科曙光实习,主要学习Openstack,云计算架构,私有云}%
  % [\begin{itemize}
  %   \item 学习Openstack各个组件,熟练使用各个基本组件命令,大致了解组件的交互过程,
  %   \item 了解云计算的board-spine-leaf组网架构
  %   \item 熟悉定制化镜像的制作流程,提出使用diskimage-builder进行镜像制作自动化,使用CI控制整个流程的概念
  % \end{itemize}]
\end{experiences}

\end{document}
