%%
%% Copyright (c) 2019-2020 ChenJiaGeng <mail@deocnf.xyz>
%% CC BY 4.0 License
%%
%% Created: 2019-04-11
%%

% Chinese version
\documentclass[zh]{resume}

% Adjust icon size (default: same size as the text)
\iconsize{\Large}

% File information shown at the footer of the last page
\fileinfo{%
  \faCopyright{} 2024--2024, 陈佳庚 \hspace{0.5em}
  \creativecommons{by}{4.0} \hspace{0.5em}
  \githublink{de-conf}{resume} \hspace{0.5em}
  \faEdit{} \today
}

\name{佳\ 庚}{陈}

\keywords{Linux, Programming, Python, C, Shell,Docker, SysAdmin, Ci,KVM}

% \tagline{\icon{\faBinoculars}} <position-to-look-for>}
% \tagline{<current-position>}

% \photo{<height>}{<filename>}

\profile{
  \mobile{18700668464}
  \email{mail@deconf.xyz}
  \blog{deconf.xyz} \\
  \university{西安石油大学}
  \degree{网络工程 \textbullet 本科}
  \birthday{1998-04-07} 
  \github{de-conf} \\
  \job{求职意向:Linux系统运维,云计算方向,Devops}
  \address{陕西省 \textbullet 西安市}
  % \icontext{<fa-briefcase>}{<text>}
  % Custom information:
  % \icontext{<icon>}{<text>}
  % \iconlink{<icon>}{<link>}{<text>}
}

\photo{64pt}{resume.jpg}
\begin{document}
\makeheader

%======================================================================
% Summary & Objectives
%======================================================================
{\onehalfspacing\hspace{2em}%
本人乐观向上,自我驱动力和学习能力强,热爱尝试新事物,认同开源和分享精神.
擅长,热衷计算机和网络技术,
有 5 年的 Linux 使用经验,掌握 Shell、Python 等脚本语言编程
\par}

%======================================================================
\sectionTitle{技能点}{\faWrench}
%======================================================================
\begin{competences}
  \comptence{操作系统}{%
    {\faLinux} Linux (5 年)---(CentOS, Ubuntu,Alpine)
  }
  \comptence{编程}{%
  {\faPython} Python,{\faTerminal} Shell, {\faDatabase} SQL
  }
  \comptence{运维}{%
  Ansible, {\faDocker}Docker
  }
  \comptence{监控}{%
    Zabbix,Prometheus,Grafana,ELK Stack
  }
  \comptence{工具}{%
    Git, Make, Tmux, Vim
  }
  \comptence{网络与安全}{%
    Wireshark,Firewalls,VPN,SSL/TLS
  }
  \comptence{数据库}{%
  Postgres,Mysql,Oracle
  }
  %\comptence{\icon{\faLanguage} 语言}{
  %  \textbf{英语} --- 读写(基础)
  %}
\end{competences}

%======================================================================
\sectionTitle{教育背景 \&\& \ \ \icon{\faBriefcase}工作经历}{\faGraduationCap}
%======================================================================
\begin{educations}
  \education%
    {2021.06}%
    [2016.09]%
    {西安石油大学}%
    {计算机学院}%
    {网络工程}%
    {本科}
\end{educations}
\begin{educations}
  \education%
    {至今}%
    [2021.6]%
    {西安}%
    {新路网络科技有限公司}%
    {运维服务部}%
    {运维工程师}
\end{educations}
%======================================================================
% \sectionTitle{主修课程}{\faGraduationCap}
%======================================================================
% \begin{itemize}
%   \item C语言程序设计基础、C++程序设计、数据结构、数据库原理与应用、操作系统
% \end{itemize}
%======================================================================
\sectionTitle{专业技能}{\faCogs}
%======================================================================
\begin{itemize}
  \item 精通Linux操作系统内核调优,numa架构调优,dpdk参数调优等.
  \item 熟悉docker容器构建,docker-compose容器编排.精通docker各类部署和运维经验.
  \item 熟悉各类zabbix,grafana监控配置与调优.
  \item 熟悉部署工具ansible. 熟悉各类CI/CD 工具,Jenkins构建pipeline脚本等
  \item 熟练shell/python脚本,可处理各类线上业务与数据.
  \item 掌握各种nginx,lvs 等web中间件配置.
  \item 掌握各类虚拟化kvm, openstack, proxmox, vmware esxi虚拟化技术与配置.
  \item 掌握postgres,mysql,oracle 主流数据库操作与优化.
  \item 了解网络设备交换路由配置.熟悉wireshark分析网络,进行网络排障与调优.
\end{itemize}

%======================================================================
\sectionTitle{工作经验}{\faCode}
%======================================================================
\begin{itemize}
  \item 近 100+ 生产环境Linux服务器日常运维工作,以及处理突发应急事件.\\ 熟练处理各类Linux发行版问题与故障. 
  \item 线上 2700+ 交换路由设备的监控与日志留存,熟悉zabbix监控模板的建立,修正与完善.\\精通zabbix 服务器与代理的维护与等.
  \item 产品线高性能网关的维护与运维. 熟悉Linux系统调优与网卡优化以及dpdk参数优化.
  \item 线上5+所高校认证系统的维护.熟悉3A流程,熟悉freeradius 配置,熟悉unlang 配置语法.
  \item 线上生产环境认证系统的部署与上线.精通docker-compose 容器编排. 熟悉云平台AWS的日常运维管理. 
  \item 内网高性能网关的研发环境配置,掌握CI/CD流程,熟悉Jenkins构建流程。
  \item 内网公共平台 confluence, jira ,gitlab平台的搭建与维护.
  \item 各类型数据包的分析,熟练定位网络故障与性能瓶颈.
  \item 内部运维知识库的管理与构建,输出相关的培训文档与手册.
\end{itemize}


%======================================================================
\sectionTitle{项目经验}{\faBriefcase}
%======================================================================
\begin{experiences}
  \experience
  []%
  {2024.03}%
  {高性能网关项目}%
  [\begin{itemize}
    \item 项目介绍:\\
    本项目主要基于\link{https://wiki.fd.io/view/VPP}{VPP} + \link{https://www.dpdk.org}{DPDK} 开源框架,构建了IPOE TO PPPOE代拨网关.
    \item 主要职责:\\
    负责项目的开发环境的搭建,性能的测试与评估,线上的维护与部署. 
    \item 主要业绩: \\
    使用systemd-coredump捕获异常崩溃,协助研发分析原因,并构建了CI/CD 流程, 实现了提交代拨后自动构建,自动打包的操作. \\
    使用 \link{https://trex-tgn.cisco.com/}{TREX} 构建各类数据包协助研发测试性能与输出指标. \\
    上线后的故障排查,日志的分析与包括分析各类数据包 pppoe会话的交互,radius 会话交互等.\\
    输出标准化部署流程文档,指导其他节点标准化的部署.并使用zabbix脚本监控运行时指标,确保运行的稳定性. 
  \end{itemize}]
  \\
  \experience
  []%
  {}%
  {高校认证项目}%
  [\begin{itemize}
    \item 项目介绍:\\
    基于\link{https://freeradius.org/}{FreeRadius} 为基础,自研计费的认证系统.
    \item 主要职责:\\
    新增学校认证系统的部署与维护,与三方认证系统的对接(电信/移动/派网).满足运营提出的认证策略(使用freeradius unlang构建的policy).
    与第三方(BRAS)网关的对接(portal 协议的对接).
    \item 主要业绩: \\
    负责的高校(西北大学,西安科技大学,西北政法大学,西安欧亚学院,湖北经济学院)认证系统500天无故障,运行平稳.
    对于私有的portal协议重新构建了\link{https://github.com/de-conf/huawei-portalv2-wireshark-pluging/blob/main/wireshark-plugin-huawei-portal-v2.lua}{wireshark解析脚本},方便同事排错.
  \end{itemize}]
  \\
  \experience
  []%
  {}%
  {出口缓存项目}%
  [\begin{itemize}
    \item 项目介绍:\\
    基于\link{https://lancache.net/}{nginx的缓存项目}的部署.减少出口带宽占用,加快内网下载速度.
    \item 主要职责: \\
    项目的选型,网络架构的设计,\link{https://github.com/lancachenet/monolithic/issues/175}{缓存卡下载问题}的排查与解决.
    \item 主要业绩: \\
    部署后节约出口带宽2Gbps,内网可达1000Mbps的下载速率,用户满意度达到95\%.
  \end{itemize}]
  \\
  \experience
  []%
  {2021.06}%
  {网络设备配置自动备份与日志留存}%
  [\begin{itemize}
    \item 项目介绍:\\
    基于Python模块Netmiko构建\link{https://github.com/de-conf/network_config_backup}{自动化脚本},在linux服务器自动备份网络设备配置,并推送私有gitlab仓库,实现了网络配置的版本控制.\\
    通过Netmiko脚本,自动配置各类网络设备syslog服务器,各学校节点配置logstash,并推送elasticsearch服务器.
    \item 主要职责: \\
    自动化脚本的构建,各类环境(gitlab,elk)的搭建与维护.
    \item 主要业绩: \\
    从无到有搭建了网络设备的配置备份与日志留存系统.满足了内部运维的需求和外部网络安全的需求.
  \end{itemize}]
  % \experience
  % [2021.05]%
  % {2021.03}%
  % {在中科曙光实习,主要学习Openstack,云计算架构,私有云}%
  % [\begin{itemize}
  %   \item 学习Openstack各个组件,熟练使用各个基本组件命令,大致了解组件的交互过程,
  %   \item 了解云计算的board-spine-leaf组网架构
  %   \item 熟悉定制化镜像的制作流程,提出使用diskimage-builder进行镜像制作自动化,使用CI控制整个流程的概念
  % \end{itemize}]
\end{experiences}

\end{document}
