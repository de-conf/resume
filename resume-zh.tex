%%
%% Copyright (c) 2019-2020 ChenJiaGeng <chenjiagen88@gmail.com>
%% CC BY 4.0 License
%%
%% Created: 2019-04-11
%%

% Chinese version
\documentclass[zh]{resume}

% Adjust icon size (default: same size as the text)
\iconsize{\Large}

% File information shown at the footer of the last page
\fileinfo{%
  \faCopyright{} 2019--2020, 陈佳庚 \hspace{0.5em}
  \creativecommons{by}{4.0} \hspace{0.5em}
  \githublink{de-conf}{resume} \hspace{0.5em}
  \faEdit{} \today
}

\name{佳\ 庚}{陈}

\keywords{Linux, Programming, Python, C, Shell,Docker, SysAdmin, Ci,KVM}

% \tagline{\icon{\faBinoculars}} <position-to-look-for>}
% \tagline{<current-position>}

% \photo{<height>}{<filename>}

\profile{
  \mobile{18700668464}
  \email{chenjiagen88@gmail.com}
  \github{de-conf} \\
  \university{西安石油大学}
  \degree{网络工程 \textbullet 本科}
  \birthday{1997-09-07}
  \texttt{求职意向:}{Linux运维开发}
  %\address{西安}
  % Custom information:
  % \icontext{<icon>}{<text>}
  % \iconlink{<icon>}{<link>}{<text>}
}

\photo{64pt}{resume.jpg}
\begin{document}
\makeheader

%======================================================================
% Summary & Objectives
%======================================================================
{\onehalfspacing\hspace{2em}%
本人乐观向上,自我驱动力和学习能力强,热爱尝试新事物,认同开源和分享精神
擅长,热衷计算机和网络技术,
有 3 年的 Linux 使用经验,基本掌握 Shell、Python 和 C 语言编程。
积极实践自由开源精神.
在 \link{https://github.com/de-conf}{GitHub} 上分享多个项目,
喜欢在个人网站\link {http://blog.deconf.xyz}{blog.deconf.xyz}记录笔记
并积极参与其他多个开源项目。
\par}

%======================================================================
\sectionTitle{技能和语言}{\faWrench}
%======================================================================
\begin{competences}
  \comptence{操作系统}{%
    \icon{\faLinux} Linux (3 年)
  }
  \comptence{编程}{%
    Python, C, Shell,SQL
  }
  \comptence{工具}{%
    SSH, Git, Make, Tmux, Vim, Ansible,Travis-Ci,KVM
  }
  %\comptence{\icon{\faLanguage} 语言}{
  %  \textbf{英语} --- 读写(基础)
  %}
\end{competences}

%======================================================================
\sectionTitle{教育背景}{\faGraduationCap}
%======================================================================
\begin{educations}
  \education%
    {2016.09}%
    [2021.06]%
    {西安石油大学}%
    {计算机学院}%
    {网络工程}%
    {本科}
\end{educations}
%======================================================================
\sectionTitle{主修课程}{\faGraduationCap}
%======================================================================
\begin{itemize}
  \item C语言程序设计基础、C++程序设计、数据结构、数据库原理与应用、操作系统
\end{itemize}
%======================================================================
\sectionTitle{计算机技能}{\faCogs}
%======================================================================
\begin{itemize}
  \item 使用Travis-Ci自动化部署博客
  \item 使用 Ansible 管理 VPS,部署个人域名邮箱、网站、Git、DNS 等服务
  \item 搭建并管理校内服务器,部署 PXE、团队Wiki 等
  \item 具有网络安全相关的基础知识与实践,参加过多次CTF攻防赛
  \item 会使用wireshark做流量分析,成功模拟了校园网认证过程
  \item 熟练使用Git代码管理工具进行团队开发和版本控制
  \item 使用Docker容器进行服务、代码部署
\end{itemize}

%======================================================================
\sectionTitle{个人项目}{\faCode}
%======================================================================
\begin{itemize}
  \item \link{https://github.com/de-conf/openwrt-dogcom}{\texttt{docgom}}
  (C)
  模拟校园网客户端拨号软件,用于嵌入式OpenWrt Linux
  \item \link{https://github.com/de-conf/blog-source}{\texttt{blog-source}}
  (Js,CSS)
  个人博客源码,使用了Travis-Ci实现了在线编辑文件,立即生效,自动部署
  \item \link{https://github.com/de-conf/resume}{\texttt{resume}}:
    (\LaTeX)
    \emph{此简历}的源文件
\end{itemize}


%======================================================================
\sectionTitle{个人经历}{\faBriefcase}
%======================================================================
\begin{experiences}
  \experience
  [2020.08]%
  {2020.10}%
  {当选网络安全协会副会长,举办西安石油大学第一届线上CTF比赛,带领新生学习CTF相关知识技能}%
  [\begin{itemize}
    \item 使用Docker部署CTF虚拟靶场,搭建比赛环境
  \end{itemize}
  ]

  \separator{0.5ex}
  \experience
  [2019.06]%
  {2020.05}%
  {已基本对个人VPS利用Ansible的各项服务实现了自动化部署 }%
  [\begin{itemize}
    \item 学习自动化运维的工具,如Ansible,Zabbix,Python
  \end{itemize}
  ]
  
  \separator{0.5ex}
  \experience
  [2018.09]%
  {2019.06}%
  {参加并获得第十二届全国大学生信息安全竞赛省级三等奖}%
  [\begin{itemize}
    \item 加入网络安全协会,和朋友一起学习CTF(网络攻防赛)相关知识,我主要负责Misc(杂项)方向相关
  \end{itemize}
  ]

 % \separator{0.5ex}
  %\experience%
   % [2018.04]%
    %{2018.08}%
    %{使用Travis-Ci为blog提供自动部署服务}%
    %[\begin{itemize}
     % \item 使用开源项目 \link{https://github.com/hexojs/hexo}{\texttt{hexo}} 和 \link{https://github.com/next-theme/hexo-theme-next}{\texttt{hexo-theme-next}} 
      %完成自己的博客网页
    %\end{itemize}]

  %\separator{0.5ex}
  %\experience%
   % [2017.08]%
   % {2017.10}%
   % {系统学习Linux基础知识,包括SHELL相关的知识}%
   % [\begin{itemize}
   %  \item 使用WireShark分析校园网登录软件流量,使用 \link{https://github.com/de-conf/openwrt-dogcom}{\texttt{dogcom}} 项目,成功在OpenWrt进行了登录拨号
   %  \item 接触嵌入式Linux(OpenWrt)
   % \end{itemize}]
\end{experiences}
\end{document}
